%% Classe du document
\documentclass[a4paper,10pt]{article}

%% Francisation
\usepackage[francais]{babel} % Indique que l'on utilise le francais
\usepackage[T1]{fontenc}
\usepackage[utf8]{inputenc} % Indique que l'on utilise tout le clavier
%\usepackage[latin1]{inputenc}

%% Réglages généraux
\usepackage[top=3cm, bottom=3cm, left=3cm, right=3cm]{geometry} % Taille de la feuille
\usepackage{lastpage}

%% Package pour le texte
\usepackage{soul} % Souligner
\usepackage{color} % Utilisation de couleurs
\usepackage{hyperref} % Créer des liens et des signets
\usepackage{eurosym}% Pour le symbole euro
\usepackage{fancyhdr}% Entête et pied de page

%% Package pour les tableaux
\usepackage{multirow} % Colonnes multiples
\usepackage{cellspace}
\usepackage{array}

%% Package pour les dessins
\usepackage{pstricks}
\usepackage{graphicx} % Importer des images
\usepackage{pdftricks} % Pour utiliser avec pdfTex
\usepackage{pst-pdf} % Pour utiliser avec pdfTex
\usepackage{pst-node} % Pose de noeuds
\usepackage{subfig}
\usepackage{float}

%% Package pour les maths
\usepackage{amsmath} % Commandes essentielles
\usepackage{amssymb} % Principaux symboles

%% Package pour le code
\usepackage{listings} % Utilisation de la couleur syntaxique des langages
\usepackage{url}


\usepackage[babel=true]{csquotes} % Permet les quotations (guillemets)
\usepackage{tocvsec2}
\usepackage{amsthm}
\usepackage{amsfonts}

\usepackage{tikz}
\usepackage{pdfpages}

\usetikzlibrary{shapes} % A revoir

%--------------------- Autres définitions ---------------------%

% Propriété des liens
\hypersetup{
colorlinks = true, % Colorise les liens
urlcolor = blue, % Couleur des hyperliens
linkcolor = black, % Couleur des liens internes
}

\definecolor{grey}{rgb}{0.95,0.95,0.95}

% Language Definitions for Turtle
%TODO: a revoir avec les couleur de gedit
\definecolor{olivegreen}{rgb}{0.2,0.8,0.5}
\definecolor{grey2}{rgb}{0.5,0.5,0.5}
\lstdefinelanguage{ttl}{
sensitive=true,
morecomment=[s][\color{grey2}]{@}{:},
morecomment=[l][\color{olivegreen}]{\#},
morecomment=[s][\color{red}]{<}{/>},
morestring=[s][\color{olivegreen}]{<http://w}{\#>},
morestring=[b][\color{blue}]{\"},
}

\lstset{
frame=single,
breaklines=true,
basicstyle=\ttfamily,
backgroundcolor=\color{grey},
basicstyle=\scriptsize,
keywordstyle=\color{blue},
commentstyle=\color{green},
stringstyle=\color{red},
identifierstyle=\color{blue}
}

%Definition de la commande pour retirer l'espace devant les ':'
\makeatletter
\@ifpackageloaded{babel}%
        {\newcommand{\nospace}[1]{{\NoAutoSpaceBeforeFDP{}#1}}}%  % !! double {{}} pour cantonner l'effet à l'argument #1 !!
        {\newcommand{\nospace}[1]{#1}}
\makeatother

\setcounter{tocdepth}{3}
%\maxsecnumdepth{subsubsection} % Dernière section numérotée

% Corps du document :
\begin{document}

% Définition des entêtes et pieds de page
\fancyhead[LE,CE,RE,LO,CO,RO]{}
\fancyfoot[LE,CE,RE,LO,CO,RO]{}
\fancyhead[LO, LE]{Interface Homme-Machine 1}
\fancyhead[RO,RE]{2012/2013}
\fancyfoot[LO,LE]{Université de \scshape{Nantes}}
\fancyfoot[RO,RE]{Page \thepage \ sur \pageref{LastPage}}
\renewcommand{\headrulewidth}{0.4pt}
\renewcommand{\footrulewidth}{0.4pt}

%\maketitle
\begin{titlepage}

\vspace*{\fill}~
\begin{center}
{\large \textsc{Rapport de Projet}} \\
\vspace{1cm}
{\LARGE Projet : Gestionnaire de listes de tâches} \\
\vspace{1cm}
COUTABLE Guillaume, RULLIER Noémie \\
\today
\end{center}
\vspace*{\fill}

\vspace{\stretch{1}}
\begin{center}
\noindent 
\includegraphics[height=2.5cm]{Images/universite.png}
\end{center}
\pagebreak
\end{titlepage}

\newpage
\tableofcontents  

% Introduction
\newpage
\pagestyle{fancy}

%%%%%%%%%%%%%%%%%%%%%%%%%%%%%%%%%%%%%%%%%%%%%%%%%%%%%%%%%%%%%%%%%%%%%%%%%%%%%
%%%%%%%%%%  Introduction générale
%%%%%%%%%%%%%%%%%%%%%%%%%%%%%%%%%%%%%%%%%%%%%%%%%%%%%%%%%%%%%%%%%%%%%%%%%%%%%

\section{Introduction}
L'objectif de ce projet fut de développer un gestionnaire avancé de tâches. Celui-ci devait permettre de créer des listes de tâches et de suivre l'avancement de celles-ci.

Afin de créer cette application que nous avons appelé \textit{Taskinator}, nous avons du établir plusieurs étapes d'avancement du projet. Ce rapport présentera ces étapes les unes après les autres (même si lors de ce projet certaines étapes se sont croisées).

% - Fonctionnalités
% - StoryBoard + PaperPrototype + Scénario d'utilisation
% - Model
% - Limites de l'application
% - Structure de IHM --> Menu widget ...
%

%%%%%%%%%%%%%%%%%%%%%%%%%%%%%%%%%%%%%%%%%%%%%%%%%%%%%%%%%%%%%%%%%%%%%%%%%%%%%
%%%%%%%%%%  Etape 1
%%%%%%%%%%%%%%%%%%%%%%%%%%%%%%%%%%%%%%%%%%%%%%%%%%%%%%%%%%%%%%%%%%%%%%%%%%%%%
\newpage
\section{Les fonctionnalités}
La première étape fut d'analyser l'ensemble des fonctionnalités que notre application devait proposer. 

\subsection{Fonctionnalités principales}
Voici dans un premier temps les fonctionnalités principales:
\paragraph{Créer une liste:} cette fonctionnalité permet à l'utilisateur de créer une liste vide.
\paragraph{Créer une liste ordonnée:} cette fonctionnalité permet à l'utilisateur de créer une liste ordonnée vide. L'ensemble des éléments de cette liste doivent être effectué dans un ordre précis.
\paragraph{Créer une tâche:} cette fonctionnalité permet à l'utilisateur de créer une tâche.
\paragraph{Supprimer un élément:} cette fonctionnalité permet de supprimer une tâche ou une liste (ordonnée ou non). Cette fonctionnalité est à manipuler avec précaution,, en effet dans le cas d'une liste, la suppression de celle-ci implique aussi la suppression de tous ses éléments (listes ou tâches).
\paragraph{Enregistrer:} cette fonctionnalité permet à l'utilisateur d'enregistrer sa liste dans un document sur son disque dur.
\paragraph{Enregistrer un template:} cette fonctionnalité permet à l'utilisateur d'enregistrer la liste qu'il vient de créer comme un template afin que la structure de celle-ci soit réutilisable.
\paragraph{Ouvrir un template:} cette fonctionnalité permet à l'utilisateur de créer une liste à partir d'un template enregistré. Il devra cependant renseigné le nom de cette liste ainsi que toutes les dates de tous les éléments. Il peut ensuite continuer à modifier cette liste.

\subsection{Fonctionnalités secondaires}
Voici les fonctionnalités secondaires:
\paragraph{Paramètre:} cette fonctionnalité permet à l'utilisateur de modifier le type de l'élément sélectionné. Il pourra par exemple choisir de modifier une liste en liste ordonnée ou en une tâche. Cette fonctionnalité est à manipuler avec précaution, en effet si l'utilisateur décide de transformer une liste en tâche l'ensemble des éléments de la liste seront supprimés.
\paragraph{Monter / Descendre:} cette fonctionnalité permet de monter ou descendre un élément dans l'arborescence de la liste. Dans le cas d'une liste, tous ces éléments sont aussi monté/descendu d'un rang. Si le changement se fait au sein d'une liste ordonnée l'ordre des éléments est aussi changé. A chaque déplacement, une vérification de la cohérence des dates est effectuée et l'utilisateur en est informé.
\paragraph{Historique:} cette fonctionnalité permet d'annuler ou rétablir des actions faites par l'utilisateur.
\paragraph{Gérer ses templates:} cette fonctionnalité permet à l'utilisateur de supprimer les templates qu'il a enregistré.

%IHM --> Mettre couleur des dates et mettre que c'est la plus récente
%%%%%%%%%%%%%%%%%%%%%%%%%%%%%%%%%%%%%%%%%%%%%%%%%%%%%%%%%%%%%%%%%%%%%%%%%%%%%
%%%%%%%%%%  Etape 2
%%%%%%%%%%%%%%%%%%%%%%%%%%%%%%%%%%%%%%%%%%%%%%%%%%%%%%%%%%%%%%%%%%%%%%%%%%%%%
\newpage
\section{StoryBoard, Scénarios, PaperPrototype}

\subsection{StoryBoard}

\subsection{PaperPrototype}

\subsection{Scénarios}
Afin de créer et tester notre PaperPrototype, nous avons créer plusieurs scénarios.
\subsubsection{Scénario 1 - Création/Utilisation/Suppression des listes}
Ce premier scénario a été utilisé pour tester le paperPrototype. Il permet de créer une liste non ordonnée et d'y ajouter une liste ordonnée avec ses propres tâches et des tâches.
\begin{enumerate}
\item{L'utilisateur choisit quel type de liste il souhaite créer, il choisit ici une liste non ordonnée. Il lui donne un nom et une date de fin. (Pour notre scénario cette liste sera appelée la liste mère)}
\item{Il créé ensuite une liste ordonnée (qui est le premier élément de la liste mère). Il lui donne un nom et une date. (Pour notre scénario cette liste sera appelée la liste 1)}
\item{Il créer ensuite une tâche qui sera le premier élément de la liste 1. Il lui donne un nom et une date. (Pour notre scénario cette tâche sera appelée la tâche 1.1)}
\item{Il créer ensuite une tâche qui sera le deuxième élément de la liste 1. Il lui donne un nom et une date. (Pour notre scénario cette tâche sera appelée la tâche 1.2)}
\item{Il souhaite maintenant échanger les tâches 1.1 et 1.2, il sélectionne donc la tâche 1.1 et la place au dessous de la tâche 1.2}
\item{Une popup peut apparaître et informe l'utilisateur que l'action qu'il vient d'effectuer provoque un conflit de date. Le champs date concerné par le conflit prend une couleur orange d'avertissement.}
\item{L'utilisateur change le type de la liste 1 en une liste non ordonnée.}
%%%%%%%% A revoir
%%%%%%%%% Est-ce que je donne des valeurs au dates --> Pour l'affichage de la popup
\item{L'utilisateur supprime la tâche 1.2}
\item{L'utilisateur supprime la tâche 1. Une popup apparaît pour l'avertir que cette suppression supprimera aussi tous les éléments de la liste.}
\end{enumerate}

\subsubsection{Scénarion 2 - Utilisation des templates}
Ce scénario permet de créer une liste à partir d'un template. De modifier cette liste et de la réenregistrer comme un nouveau template.
\begin{enumerate}
\item{L'utilisateur choisit de créer une liste à partir d'un template. Il choisit ici le template qui correspond à la préparation d'un cours, puis donne un nom et une date à sa liste.}
\item{Il va ensuite pour tous les éléments de ce template donner une date.}
\item{L'utilisateur va ensuite modifier cette liste en y ajoutant une tâche à la liste mère (il lui donne un nom et une date).}
\item{Il va ensuite vouloir enregistrer cette nouvelle liste comme un nouveau template.}
%% Est-ce que quand il charge une liste à partir d'un template quand il va enregister --> creéer une nouvelle liste comme normalement
% mais s'il fait enregister un template est-ce que ça lui propose de modifier le template qu'il a charger ou d'en créer un nouveau ou est-ce qu'il peut l'écraser tout simplement en lui donnant le même nom ?
\end{enumerate}


%%%%%%%%%%%%%%%%%%%%%%%%%%%%%%%%%%%%%%%%%%%%%%%%%%%%%%%%%%%%%%%%%%%%%%%%%%%%%
%%%%%%%%%%  Etape 3
%%%%%%%%%%%%%%%%%%%%%%%%%%%%%%%%%%%%%%%%%%%%%%%%%%%%%%%%%%%%%%%%%%%%%%%%%%%%%
\newpage
\section{}


%%%%%%%%%%%%%%%%%%%%%%%%%%%%%%%%%%%%%%%%%%%%%%%%%%%%%%%%%%%%%%%%%%%%%%%%%%%%%
%%%%%%%%%%  Etape 4
%%%%%%%%%%%%%%%%%%%%%%%%%%%%%%%%%%%%%%%%%%%%%%%%%%%%%%%%%%%%%%%%%%%%%%%%%%%%%
\newpage

%%%%%%%%%%%%%%%%%%%%%%%%%%%%%%%%%%%%%%%%%%%%%%%%%%%%%%%%%%%%%%%%%%%%%%%%%%%%%
%%%%%%%%%%  CONCLUSION GENERALE
%%%%%%%%%%%%%%%%%%%%%%%%%%%%%%%%%%%%%%%%%%%%%%%%%%%%%%%%%%%%%%%%%%%%%%%%%%%%%
\newpage
\section{Conclusion générale}
Ce projet nous a permis de contribuer au Web Sémantique en apportant notre propre lot de données ouvertes prêtes à être utilisées.
Ce travail nous a montré que chacun pouvait contribuer au Cloud of Linked Data et participer de façon à obtenir une base de données géante et présente sur le web.
\end{document}


\begin{figure}[H]
    \center
    \includegraphics[width=15cm]{Images/Csv.png}
    \caption{Extrait des données récupérées concernant les adresses postales de Nantes Métropole}
\end{figure}
